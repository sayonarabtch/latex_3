\documentclass{article}
\usepackage[utf8]{inputenc}
\usepackage[russian]{babel}
\usepackage{amsmath}
\usepackage{hyperref}
\usepackage{amssymb}
\numberwithin{equation}{section}

\begin{document}
    
    \begin{center}
        {\Large МОСКОВСКИЙ ГОСУДАРСТВЕННЫЙ УНИВЕРСИТЕТ\\
        имени М.В. ЛОМОНОСОВА\\
        МЕХАНИКО-МАТЕМАТИЧЕСКИЙ ФАКУЛЬТЕТ}
    \end{center} 
    
    \vspace{4cm}
    \begin{center}
        {\LARGE {\bf A. В. Чашкин}}
    \end{center}
    
    \begin{center}
        {\LARGE \bf{ЛЕКЦИИ\\
        ПО ДИСКРЕТНОЙ МАТЕМАТИКЕ}}
    \end{center}
    
    \begin{center}
         {\bf Учебное пособие}
    \end{center}
    \newpage
    
    \tableofcontents
    \newpage
    
    \section{Теорема Холла}
    Произвольное подмножество попарно несмежных ребер графа
    $G$ называется его \emph{паросочетанием}. Паросочетание называется \emph{совершенным}, если каждая вершина графа инцидентна какому-нибудь ребру паросочетания.Граф
    $G_1$ на рис. 5.2 обладает совершенным паросочетанием, например таким — $\{\{v_1, v_4\}, \{v_2, v_5\}, \{v_3, v_6\}\}$, в то время как ни в одном из графов на рис. 5.3 совершенного паросочетания нет.\\
    Будем говорить, что паросочетание $M$ двудольного графа$ G = (V_1, V_2, E)$ покрывает долю $V_i$, если каждая вершина этой доли инцидентна какому-нибудь ребру этого паросочетания. Пусть $A$ — подмножество доли $V_1$ графа $G = (V_1, V_2, E)$. Тенью $S(A)$ множества $A$ называется подмножество вершин доли $V_2$, каждая из которых смежна хотя бы с одной вершиной из $A$. В следующей теореме Ф. Холла устанавливается необходимое и достаточноеусловие существования паросочетания. \\
    
    \textbf{Теорема } {\it Конечный двудольный граф $G$ с долями $V_!$ и $V_2$ обладает паросочетанием, покрывающим долю $V_1$, тогда и только тогда, когда $ |S(X)| \ge |X| $ для любого непустого $X \subseteq V_1 .$}\\
    
    {\scshape Доказательство}. Необходимость условия теоремы очевидна, так как если тень некоторого подмножества X состоит из меньшего чем само $X$ числа вершин, то междувершинами из $X$ и вершинами из его тени нельзя установить взаимно однозначное соответствие.Установим достаточность условия теоремы. Сделаем это индукцией почислу вершинn в $V_1$. В основание индукции положим очевидный случай $n = 1$. Допустим, что достаточность доказана при всех значениях $|V1|$ непревосходящих $m − 1$. Пусть $G = (V_1, V_2, E)$ — произвольный двудольныйграф, в котором доля $V_1$ состоит из $m$ вершин. Рассмотрим два возможныхслучая.\\
    ($i$) Для любого непусттого подмножества $A ⊂ V1$ справедливо строгое неравенство $|S(A)| > |A|$ $V_1$ и $V_2$ выберем по одной вершине, которые связаны ребром e, и удалим эти вершины со всеми инцидентными им ребрами из $G$. В новом графе $G^{'} = ( V_1^{'} , V_2^{'} , E^{'} )$ доля $V_1^'$ состоит из $m - 1$ вершин и для каждого непустого подмножества $X \subseteq V_1^'$ справедли- во неравенство $|S(X)|\ge| X|$, так как после удаления ребра e размер тени любого подмножества из $V_1$ уменьшился не более чем на единицу. По пред- положению индукции в $G^{'}$ существует паросочетание $M^{'}$, покрывающее $V_1^{'}$ . Тогда объединение $M^{'}$ с ребром $e$ будет требуемым паросочетанием в $G$.\\
    ($ii$) В $V_1$ найдется такое подмножество $A$, что $|S(A)| = |A|$. По предположению индукции в двудольном графе $ G_A = (A, S(A), E_A)$, долями которого являются множества $A$ и $S(A)$, а ребрами — все ребра графа $G$ инцидентные вершинам этих множеств, существует совершенное паросочетание $M_A$. Рассмотрим двудольный граф $G^{'} = ( V_1^{'} , V_2^{'} , E^{'} )$, где $V  = V1 \ A, V  V2 \ S(A)$, $E_{'}$ — множество ребер инцидентных одновременно $V_1^{'}$ и $V_2^{'}$. Если для какого-нибудь $B \subset V_1$ выполняется неравенство $|S(B)| < |B|$, то в силу того, что $A$ и $B$ не пересекаются, справедливы неравенства\\
    
    |S(A \cup B)| \le |S(A)| + |S(B)| = |A| + |S(B)| < |A| + |B| = |A \cup B|,\\
    
    которые очевидно противоречат условию теоремы. Следовательно, $|S(B)| \ge |B|$ для любого непустого $B \subseteq V_1^{'}$ . Поэтомув силуиндуктивного предположения в графе $G^{'}$ существует паросочетание $M^{'}$, покрывающее долю $V_1^{'}$ . Легко видеть, что объединение $M_A$ и $M^{'}$ будет паросочетанием в графе $G$, и это паросочетание будет покрывать $V_1$. Теорема доказана.
\end{document}    
    